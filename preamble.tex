% preamble.tex

%%%%%%%%%%%%%%% Chinese %%%%%%%%%%%%%%%
%%%%% Uncomment the following lines to use Chinese
% \usepackage{xeCJK}
% \usepackage{fontspec}
%%%%%%%%%%%%%%% Chinese %%%%%%%%%%%%%%%

%%%%%%%%%%%%%%% Theme %%%%%%%%%%%%%%%
\usetheme{CambridgeUS} % try Madrid
\usecolortheme{beaver} % try beaver, dolphin, seahorse
% \usefonttheme[onlymath]{serif} % try "professionalfonts"
\usefonttheme{serif}  % standard font (same with that in ``standalone'')
% \setCJKmainfont{Microsoft YaHei} % try SimSun
% \setmainfont{Gill Sans}
% \setsansfont{Gill Sans}

\usepackage[export]{adjustbox}

\setbeamersize{text margin left = 2em, text margin right = 1em}
\setbeamercolor{footnote mark}{fg = teal}
\setbeamertemplate{itemize items}[default]
\setbeamertemplate{enumerate items}[default]

\renewcommand*{\thefootnote}{\alph{footnote}}
%%%%%%%%%%%%%%% Theme %%%%%%%%%%%%%%%

\usepackage{graphicx, subcaption}

\usepackage{amssymb, pifont}
\newcommand{\cmark}{\ding{51}}
\newcommand{\xmark}{\ding{55}}

%%%%%%%%%%%%%%% Algorithms %%%%%%%%%%%%%%%
\usepackage{algorithm}
\usepackage[noend]{algpseudocode}
% for two-column algorithms
% see https://tex.stackexchange.com/a/159431.
\usepackage{varwidth}

\newcommand{\code}[2]{#1:#2}
\newcommand{\hStatex}[0]{\vspace{4pt}}
\renewcommand\algorithmicthen{}
\renewcommand\algorithmicdo{}

\algblockdefx[upon]{When}{EndWhen}%
  [3][Unknown]{\textbf{when for every} #3 \textbf{received a quorum of} #1(#2)}%
  {\textbf{end}}

\algblockdefx[upon]{WhenReceive}{EndWhenReceive}%
  [1][Unknown]{\textbf{when received} #1}%
  {\textbf{end}}

\algblockdefx[upon]{Upon}{EndUpon}%
  [1][Unknown]{\textbf{upon} #1}%
  {\textbf{end upon}}

% Remove the end blank line when using noend in conjunction with custom blocks
% See https://tex.stackexchange.com/a/172065
\makeatletter
\ifthenelse{\equal{\ALG@noend}{t}}%
  {\algtext*{EndWhen}}
  {}%
\makeatother

\makeatletter
\ifthenelse{\equal{\ALG@noend}{t}}%
  {\algtext*{EndWhenReceive}}
  {}%
\makeatother

\makeatletter
\ifthenelse{\equal{\ALG@noend}{t}}%
  {\algtext*{EndUpon}}
  {}%
\makeatother
%%%%%%%%%%%%%%% Algorithms %%%%%%%%%%%%%%%

%%%%%%%%%%%%%%% Theorems %%%%%%%%%%%%%%%
% theorems (global numbering)
\theoremstyle{definition}
\newtheorem{property}[theorem]{Property}
\newtheorem{assumption}{\textsc{Assumption}}
%%%%%%%%%%%%%%% Theorems %%%%%%%%%%%%%%%

\usepackage{caption}
\DeclareCaptionLabelSeparator{none}{}
\captionsetup{labelsep = none}

\makeatletter
\let\@@magyar@captionfix\relax
\makeatother

%%%%%%%%%%%%%%% Tables %%%%%%%%%%%%%%%
\newcommand{\incell}[2]{\begin{tabular}[c]{@{}c@{}}#1\\[3pt] #2\end{tabular}}
%%%%%%%%%%%%%%% Tables %%%%%%%%%%%%%%%

%%%%%%%%%%%%%%% Figures %%%%%%%%%%%%%%%
% for fig without caption: #1: width/size; #2: fig file
\newcommand{\fig}[2]{
  \begin{figure}[htp]
    \centering
    \includegraphics[#1]{#2}
  \end{figure}
}

% for fig with caption: #1: width/size; #2: fig file; #3: fig caption
\newcommand{\figcap}[3]{
  \begin{figure}[htp]
    \centering
    \includegraphics[#1]{#2}
    \caption{#3}
  \end{figure}
}

%%%%% For Chinese %%%%%
% \renewcommand\figurename{图:\;}
% \renewcommand\tablename{表:\;}
%%%%% For Chinese %%%%%
%%%%%%%%%%%%%%% Figures %%%%%%%%%%%%%%%

%%%%%%%%%%%%%%% TiKZ %%%%%%%%%%%%%%%
\usepackage[linewidth = 1pt, framemethod = TikZ]{mdframed}
\mdfsetup{frametitlealignment=\center}
%%%%%%%%%%%%%%% TiKZ %%%%%%%%%%%%%%%

%%%%%%%%%%%%%%% Citation %%%%%%%%%%%%%%%
% try "backend = bibtex", "backend = biber"
\usepackage[natbib = true, backend = biber, style = authoryear, maxbibnames = 99]{biblatex}
\setbeamertemplate{bibliography item}[article]
\renewcommand*{\bibfont}{\footnotesize}
% \addbibresource{xxx.bib}

\newcommand{\ncite}[1]{\violet{\footnotesize [\cite{#1}]}}
%%%%%%%%%%%%%%% Citation %%%%%%%%%%%%%%%

%%%%%%%%%%%%%%% Colors %%%%%%%%%%%%%%%
\usepackage{xcolor}

\definecolor{DarkRed}{rgb}{0.55, 0.0, 0.0}

\newcommand{\red}[1]{\textcolor{red}{#1}}
\newcommand{\green}[1]{\textcolor{green}{#1}}
\newcommand{\blue}[1]{\textcolor{blue}{#1}}
\newcommand{\purple}[1]{\textcolor{purple}{#1}}
\newcommand{\cyan}[1]{\textcolor{cyan}{#1}}
\newcommand{\violet}[1]{\textcolor{violet}{#1}}
\newcommand{\lgray}[1]{\textcolor{lightgray}{#1}}
\newcommand{\teal}[1]{\textcolor{teal}{#1}}
\newcommand{\brown}[1]{\textcolor{brown}{#1}}
\newcommand{\orange}[1]{\textcolor{orange}{#1}}

\newcommand{\hl}[2]{\fcolorbox{#1}{#1!60}{#2}}
%%%%%%%%%%%%%%% Colors %%%%%%%%%%%%%%%

%%%%%%%%%%%%%%% thankyou %%%%%%%%%%%%%%%
\newcommand{\thankyou}{
  \begin{frame}[noframenumbering]
    \begin{center}
      \fig{width = 0.618\textwidth}{figs/thankyou}
      \vspace{0.30cm}
      Hengfeng Wei (hfwei@nju.edu.cn)
    \end{center}
  \end{frame}
}
%%%%%%%%%%%%%%% thankyou %%%%%%%%%%%%%%%